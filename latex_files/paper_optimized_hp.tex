\documentclass[nofootinbib]{revtex4-2}

%%%%%%%%%%%%%%%%%%%%%%%%%%%%

\usepackage{scrextend}
\usepackage{graphicx}
\usepackage{indentfirst}
\usepackage{latexsym} 
\usepackage{multirow}
\usepackage{epsfig}
\usepackage{hyperref}
\usepackage{color}
\usepackage[usenames,dvipsnames]{xcolor}
\usepackage{amssymb}
\usepackage{amsfonts}
\usepackage{amsmath}
\usepackage{bm}
\usepackage{mathrsfs}
\usepackage{cleveref}
\usepackage{algorithm}
\usepackage{algorithmic}
\usepackage{subcaption}

\algsetup{indent=2em}
\renewcommand{\algorithmicrequire}{\textbf{input:}}
\renewcommand{\algorithmicensure}{\textbf{output:}}
\renewcommand{\algorithmicforall}{\textbf{for each}}

\usepackage[normalem]{ulem}
\useunder{\uline}{\ul}{}

%%%%%%%%%%%%%%%%%%%%%%%%%%%%

%------ comments ----- %
\newcounter{comment}
\newenvironment{comment}[1][]{\refstepcounter{comment}\par\medskip\noindent%
   \textbf{Comment~\thecomment. #1} \rmfamily}{\medskip}
%------------------------------%

\newcommand{\algword}[1]{{\ \mathbf{#1}\ }}

\makeatletter
\newcommand{\multiline}[1]{%
  \begin{tabularx}{\dimexpr\linewidth-\ALG@thistlm}[t]{@{}X@{}}
    #1
  \end{tabularx}
}
\newcommand{\bea}{\begin{eqnarray}}
\newcommand{\eea}{\end{eqnarray}}
\makeatother

% Calligraphy letters
\def\cA{ {\cal A} }
\def\cB{ {\cal B} }
\def\cC{ {\cal C} }
\def\cE{ {\cal E} }

\def\cF{ {\cal F} }
\def\cH{ {\cal H} }
\def\cK{ {\cal K} }
\def\cP{ {\cal P} }

\def\cQ{ {\cal Q} }
\def\cR{ {\cal R} }
\def\cS{ {\cal S} }
\def\cT{ {\cal T} }

\def\cO{ {\cal O} }

\newcommand{\TSrb}{TS$_{\text{RB}}$}
\newcommand{\TSreg}{TS$_{\text{reg}}$}

\newcommand{\hLS}{$h_{\text{LS}}$}
\newcommand{\hNR}{$h_{\text{NR}}$}
\newcommand{\hML}{$h_{\text{ML}}$}


\newcommand{\citeme}{{\color{orange} CITE ME!}}

\newcommand{\mt}[1]{{\color{blue}~\textsf{[(MT) #1]}}}
\newcommand{\fc}[1]{{\color{violet}~\textsf{[(FC) #1]}}}
\newcommand{\adp}[1]{{\color{green}~\textsf{[(ADP) #1]}}}
\newcommand{\AV}[1]{{\color{red}~\textsf{[(AV) #1]}}}



% complex, real, integer, etc numbers
\def\natural{\mathbb{N}}
\def\integer{\mathbb{Z}}
\def\real{\mathbb{R}}
\def\complex{\mathbb{C}}

\newcommand{\mb}[1]{\mbox{\boldmath $#1$}}
\newcommand{\cI}{{\cal I}}
\newcommand{\be}{\begin{equation}}
\newcommand{\ee}{\end{equation}}
\newcommand{\blank}{\bigskip}
\newcommand{\lam}{\lambda}
\newcommand{\Lam}{\Lambda}
\newcommand{\ka}{\kappa}
\newcommand{\ft}{\footnote}
\newcommand{\mlam}{\mb{\lambda}}
\newcommand{\bigo}[1]{{\cal O}({#1})}

\newcommand{\code}[1]{\mintinline{Python}{#1}}

\interfootnotelinepenalty=10000


\def\addFaMAF{Facultad de Matem\'atica, Astronom\'ia, F\'isica y Computaci\'on,\\
 Universidad Nacional de C\'ordoba, (5000) C\'ordoba, Argentina}

\def\addCONICET{CONICET}

\def\addIsistan{ISISTAN-CONICET Research Institute, \\UNICEN University, (7000) Tandil, Buenos Aires, Argentina}

%%%%%%%%%%%%%%%%%%%%%%%%%%%%


\begin{document}

\title{Gravitational waves reduced basis with an optimized hp-greedy refinement}

\author{Franco Cerino}
\affiliation{\addFaMAF}
\affiliation{\addCONICET}

\author{J. Andrés Diaz-Pace}
\affiliation{\addIsistan}

\author{Manuel Tiglio}
\affiliation{\addFaMAF}
\affiliation{\addCONICET}

\author{Atuel Villegas}
\affiliation{\addFaMAF}

\begin{abstract}

\end{abstract}

\maketitle

\fc{revisar que afiliación corresponde para Atuel}
%-----------------------------------------------------------------------------------
\section{Introduction} \label{sec:intro}
%-----------------------------------------------------------------------------------

\fc{agregando algunas ideas para la introducción:}

 
Parameter estimation of the source of gravitational waves is a main topic related to LIGO observations, where the interest is to infer the properties of the black holes or neutron stars involved. In this field of research, speeding up parameter estimation can provide the possibility to measure the electromagnetic couterpart of gravitational waves\cite{Morisaki_2020}. A main bottleneck in this task is the computational cost of computing surrogate models of gravitational waves to perform algorithms as Markov-Chain Monte Carlo (MCMC). The standard approach to building surrogate models is to use a global reduced basis to represent a parameterized set of gravitational waves, which is built in a region close to the parameters found in the trigger part of the detection pipeline and the basis must be accurate and low dimension to enable fast predictions.
 
In this work, we propose to partition the parameter space in an optimized way to obtain a set of accurate and low-dimensional bases in a divide-and-conquer approach. The fundament for doing this is that smaller parameter spaces imply lower complexity, resulting in a set of reduced bases with a lower dimension than a global one, implying a faster prediction time. We use the hp-greedy method \cite{EftangThesis}, which is a methodology that adaptively partitions the parameter space and builds local reduced basis, which can be searched in a fast way given a tree structure built while training the model. This approach has proven the feasibility of speed improvements in gravitational wave surrogates \cite{cerino2022automated}, obtaining a set of reduced basis with less dimension and no loss of accuracy than a global one.
 
hp-greedy is a more complex approach than the standard reduced basis method, that introduces new hyperparameters, where the resulting models are very sensitive to their values. Thus, they need to be chosen carefully to obtain a good representation of the space of interest. Also, more computing is needed to find the best representation, because for each set of hyperparameters a new set of basis needs to be built and tested to address its properties. In machine learning, results demonstrate that the challenge of hyperparameter optimization is a direct impediment to scientific progress. This case is not an exception and requires a rigorous and efficient search for the hp-greedy approach. In this paper, we use a bayesian approach that can do an efficient and effective search to obtain a set of reduced basis that describe in a precise and succinct way the function space of interest, in this case comprised by gravitational waves.
 

%-----------------------------------------------------------------------------------
\section{hp-gredy Refinement} \label{sec:hp}
%-----------------------------------------------------------------------------------
Brief description of the method \cite{cerino2022automated,EftangThesis} and its hyperparameters. 

\fc{creo que no es necesario explicar hp-greedy al nivel más bajo posible, referenciar a \cite{cerino2022automated,EftangThesis}}

%-----------------------------------------------------------------------------------
\section{Hyperparameter optimization} \label{sec:bay}
%-----------------------------------------------------------------------------------
\fc{Introducir búsqueda standard de hiperparámetros y luego el método bayesiano}
%-----------------------------------------------------------------------------------
\section{Optimized reduced basis for gravitational waves} \label{sec:results}
%-----------------------------------------------------------------------------------

\subsection{Dataset} \label{sec:dataset}


\subsection{Results}
\fc{bayesian approach vs otros métodos. mejora de tiempo offline.}
\fc{mejora de tiempo prediccion sin perdida de precision. mejora de tiempo online}

%-----------------------------------------------------------------------------------
\section{Discussion} \label{sec:com}
%-----------------------------------------------------------------------------------

%-----------------------------------------------------------------------------------
\section{Acknowledgments} \label{sec:ack}
%-----------------------------------------------------------------------------------


\bibliography{references}

\end{document}
